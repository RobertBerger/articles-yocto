%This is nor for dummies, but for mere m0rtals
%%This is a very basic article template.
%%There is just one section and two subsections.
%\documentclass{article}
\documentclass[a4paper,11pt]{article}
%\documentclass[letterpaper,11pt]{article}
%\usepackage{sidecap}
\usepackage[lofdepth,lotdepth]{subfig}
\usepackage{ccicons}
\usepackage{url}
\usepackage{hyperref}
\usepackage{breakurl}

\usepackage{multicol}
%\usepackage{times}
\usepackage{charter}    % Charter
\usepackage{courier}    % Courier, \texttt only
\usepackage{float}      % floating graphics 
\usepackage{graphicx}
\usepackage{wrapfig}
%\usepackage{hyperref}
\usepackage{listings}
\hypersetup{colorlinks=false}
\hypersetup{
    colorlinks,%
    citecolor=black,%
    filecolor=black,%
    linkcolor=black,%
    urlcolor=black,%
    breaklinks=true,%
}

% Quite good monospace-font:
\renewcommand{\ttdefault}{txtt}


% box for Verbatim
\usepackage{verbatim}
\usepackage{framed}


\usepackage[top=2cm, bottom=2cm, left=2cm, right=2cm]{geometry} 
\usepackage{sectsty}
\sectionfont{\Large}

\usepackage{color}
\usepackage{xcolor}

\usepackage{pstricks}

\usepackage{caption}
\DeclareCaptionFont{white}{\color{white}}
\DeclareCaptionFormat{listing}{\colorbox{gray}{\parbox{\textwidth}{#1#2#3}}}
\captionsetup[lstlisting]{format=listing,labelfont=white,textfont=white}


%\lstset{
%numbers=left, 
%stepnumber=1, 
%frame=single, 
%breaklines=true,
%breakatwhitespace=false, 
%frame=shadowbox, 
%rulesepcolor=\color{black},
%basicstyle = \ttfamily,
%columns=fullflexible
%}

\lstset{
         %basicstyle=\footnotesize\ttfamily, % Standardschrift
         basicstyle=\small\ttfamily,
         %basicstyle=\small\sffamily,
         numbers=left,               % Ort der Zeilennummern
         numberstyle=\tiny,          % Stil der Zeilennummern
         %stepnumber=2,               % Abstand zwischen den Zeilennummern
         numbersep=5pt,              % Abstand der Nummern zum Text
         tabsize=2,                  % Groesse von Tabs
         extendedchars=true,         %
         breaklines=true,            % Zeilen werden Umgebrochen
         breakatwhitespace=false,
         keywordstyle=\color{red},
                frame=b,
 %        keywordstyle=[1]\textbf,    % Stil der Keywords
 %        keywordstyle=[2]\textbf,    %
 %        keywordstyle=[3]\textbf,    %
 %        keywordstyle=[4]\textbf,   \sqrt{\sqrt{}} %
         stringstyle=\color{white}\ttfamily, % Farbe der String
         showspaces=false,           % Leerzeichen anzeigen ?
         showtabs=false,             % Tabs anzeigen ?
         xleftmargin=17pt,
         framexleftmargin=17pt,
         framexrightmargin=5pt,
         framexbottommargin=4pt,
         %backgroundcolor=\color{lightgray},
         showstringspaces=false,      % Leerzeichen in Strings anzeigen ?        
         frame=bottomline,
%         escapeinside={(*@}{@*)}
         escapeinside={(*->}{<-*)}
 }

\long\def\greybox#1{%
    \newbox\contentbox%
    \newbox\bkgdbox%
    \setbox\contentbox\hbox to \hsize{%
        \vtop{
            \kern\columnsep
            \hbox to \hsize{%
                \kern\columnsep%
                \advance\hsize by -2\columnsep%
                \setlength{\textwidth}{\hsize}%
                \vbox{
                    \parskip=\baselineskip
                    \parindent=0bp
                    #1
                }%
                \kern\columnsep%
            }%
            \kern\columnsep%
        }%
    }%
    \setbox\bkgdbox\vbox{
        \pdfliteral{0.85 0.85 0.85 rg}
        \hrule width  \wd\contentbox %
               height \ht\contentbox %
               depth  \dp\contentbox
        \pdfliteral{0 0 0 rg}
    }%
    \wd\bkgdbox=0bp%
    \vbox{\hbox to \hsize{\box\bkgdbox\box\contentbox}}%
    \vskip\baselineskip%
}

%\newcommand{\ScriptsDir}{../../../../eldk-5-0-scripts}
%\newcommand{\DenxDir}{../../../../eldk-5-0-denx}

